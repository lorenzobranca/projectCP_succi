% !TeX spellcheck = it_IT
%Fronte-retro, Nuovo capitolo a destra
\documentclass [11pt]{article}
%\usepackage{geometry}
%Lingua
%\usepackage[italian]{babel}
%\selectlanguage{italian}

%Per i caratteri
\usepackage[utf8]{inputenc}
\usepackage[T1]{fontenc}


%Per la lista dei simboli
\usepackage{nomencl}
\makenomenclature

\renewcommand{\nomname}{List of Symbols}


%Per citare con \hiperref, inserire url e altro
\usepackage{hyperref}

%Pacchetti per i simboli
\usepackage{amssymb}

%Pacchetto per le formule matematiche
\usepackage{amsmath}



%Per i grafici
\usepackage{graphicx}


%Per i caratteri con mathcal, disattivato
%\usepackage{eucal}

%Per i teoremi
\usepackage{amsthm}


\usepackage{chngpage}

\usepackage{color} %red, green, blue, yellow, cyan, magenta, black, white
\definecolor{mygreen}{RGB}{28,172,0} % color values Red, Green, Blue
\definecolor{mylilas}{RGB}{170,55,241}
\definecolor{mygray}{RGB}{235,235,235}



\usepackage{bigfoot} % to allow verbatim in footnote
\usepackage[numbered,framed]{matlab-prettifier}
\usepackage{filecontents}



\usepackage{eso-pic}
\usepackage{array,multirow,graphicx}
\usepackage{pgfplots}
\usepackage{multicol}
\usepackage{amsfonts}
\usepackage{fancyhdr}
\usepackage{tikz}
\usepackage{amsopn}
\usepackage{algorithm}
\usepackage[noend]{algpseudocode}
\usepackage{setspace}
\usepackage{caption}
\captionsetup{margin=10pt, font=footnotesize, labelfont=bf}
\usepackage[nottoc]{tocbibind}
\usepackage{mathtools}




%Comodità
\newcommand{\ppi}{\mathbf{\pi}}
\renewcommand{\epsilon}{\varepsilon}
\renewcommand{\phi}{\varphi}
\newcommand{\bu}{\mathbf{1}}
\newcommand{\unosu}[1]{\frac{1}{#1}}
\renewcommand{\l}{\lambda}
\renewcommand{\L}{\Lambda}
\newcommand{\tA}{\widetilde{A}}
\newcommand{\tB}{\widetilde{B}}
\newcommand{\tC}{\widetilde{C}}
\newcommand{\tE}{\widetilde{E}}
\newcommand{\tP}{\widetilde{P}}
\newcommand{\tG}{\widetilde{G}}
\newcommand{\ta}{\tilde{a}}
\newcommand{\tv}{\tilde{v}}


%Parentesi
\newcommand{\virgolette}[1]{``#1''}
\newcommand{\br}[1]{\<#1\>}
\newcommand{\tonde}[1]{\left( #1 \right)}
\newcommand{\quadre}[1]{\left[ #1 \right]}
\newcommand{\graffe}[1]{\left\{ #1 \right\}}
\newcommand{\assoluto}[1]{\left| #1 \right|}
\newcommand{\norma}[1]{\left\| #1 \right\|}

%Norme
\newcommand{\normaW}[1]{\norma{#1}_{\W}}
\newcommand{\normaWu}[1]{\norma{#1}_{\W_1}}
\newcommand{\normaF}[1]{\norma{#1}_{\F}}
\newcommand{\normaQT}[1]{\norma{#1}_{\QT}}
\newcommand{\normaGQT}[1]{\norma{#1}_{\GQT}}
\newcommand{\normalp}[1]{\norma{#1}_{\ell^p}}
\newcommand{\normalinf}[1]{\norma{#1}_{\ell^\infty}}
\newcommand{\normalu}[1]{\norma{#1}_{\ell^1}}
\newcommand{\normald}[1]{\norma{#1}_{\ell^2}}
\newcommand{\normap}[1]{\norma{#1}_{p}}
\newcommand{\normauno}[1]{\norma{#1}_{1}}
\newcommand{\normainf}[1]{\norma{#1}_{\infty}}


%operatori
\newcommand{\minimo}[1]{\min \{ #1 \}}
\DeclareMathOperator{\diag}{diag}
\DeclareMathOperator{\trunc}{trunc}
\DeclareMathOperator{\trid}{trid}



%mathbbs
\newcommand{\N}{\mathbb{N}}
\newcommand{\Z}{\mathbb{Z}}
\newcommand{\R}{\mathbb{R}}
\newcommand{\C}{\mathbb{C}}

\newcommand{\D}{\mathbb{D}}


\newcommand{\A}{\mathbb{A}}
\newcommand{\BB}{\mathbb{B}}
\newcommand{\T}{\mathbb{T}}

%mathbfs
\renewcommand{\P}{\mathbf{P}}
\newcommand{\E}{\mathbf{E}}


%mathcals
\newcommand{\W}{\mathcal{W}}
\newcommand{\F}{\mathcal{F}}
\newcommand{\QT}{\mathcal{QT}}
\newcommand{\GQT}{\mathcal{GQT}}
\newcommand{\B}{\mathcal{B}}
\renewcommand{\S}{\mathcal{S}}
\newcommand{\EE}{\mathcal{E}}


%bold symbols
\newcommand{\uni}{\bold 1}
\newcommand{\boldpi}{\boldsymbol{\pi}}
\newcommand{\bp}{\boldsymbol{\pi}}
\newcommand{\xx}{\mathbf{x}}
\newcommand{\bb}{\mathbf{b}}
\renewcommand{\i}{\mathbf{i}}
\newcommand{\X}{\mathbf{X}}
\newcommand{\Y}{\mathbf{Y}}




\newtheorem{thm}{Theorem}[section]

\theoremstyle{definition}
\newtheorem{rem}{Remark}

\newtheorem{prop}[thm]{Proposition}
\newtheorem{lem}[thm]{Lemma}
\newtheorem{cor}[thm]{Corollary}

\theoremstyle{definition}
\newtheorem{defn}{Definition}




\newcommand{ \ssss }[1]{\subsubsection*{#1}}
\newcommand{ \sss }[1]{\subsection*{#1}}

\begin{document}
\sss{Equazione}
%
Discretizzazione con differenze finite per l'equazione
%
$$
	\partial_t \phi(t,x)+V(t,x)\nabla \phi(t,x)=D(t,x)\nabla^2 \phi(t,x) + 						a(x)\phi(t,x)-b\phi(t,x)^2.
$$	
%
Supponiamo $\phi:[0,T]\times [0,L]\to \R$, quindi $x\in[0,L]$ ha dimensione $1$. L'equazione diventa
%
$$
\partial_t \phi(t,x)+V(t,x)\partial_x \phi(t,x)=D(t,x)\partial_{xx} \phi(t,x) + 						a(x)\phi(t,x)-b\phi(t,x)^2.
$$	




\sss{Griglie}

Discretizziamo $[0,T]$ con passo $\Delta t$ e $[0,L]$	con passo $\Delta x$. Otteniamo due griglie di punti: $t_0,t_1,\dots,t_N$, $x_0,x_1,\dots x_M$. Più precisamente
%
\begin{align*}
	t_n&=n\Delta t,\quad n=0,\dots,N,\\ 
	x_j&=j\Delta x,\quad j=0,\dots,M.
\end{align*}
%	
Cerchiamo le seguenti approssimazioni:
%
\begin{align*}
\phi^n_j\approx \phi(t_n,x_j),\qquad n=0,\dots,N,\quad  j=0,\dots,M.	
\end{align*}	
%



\sss{Discretizzazione degli operatori differenziali}
Applichiamo le seguenti discretizzazioni locali.
\begin{itemize}
%	
	\item Differenza in avanti per le derivate del primo ordine:
%
	\begin{align*}
		\partial_t\phi (t_n,x_j)&\approx \frac{\phi(t_{n+1},x_j)-\phi(t_n,x_j) }{\Delta t},\\
		\partial_x\phi (t_n,x_j)&\approx \frac{\phi(t_n,x_{j+1})-\phi(t_n,x_j) }{\Delta x}.
	\end{align*}
%
	\item Discretizzazione con stencil $[1,-2,1]$ per la derivata seconda:
%
	\begin{align*}
	\partial_{xx}\phi(t_n,x_j)\approx \frac{\phi(t_n,x_{j+1})-2\phi(t_subn,x_j)+\phi(t_n,x_{j-1})}{\Delta x^2}.
	\end{align*}
%	
\end{itemize}
%

\newpage

\sss{Schema numerico}



Vogliamo ottenere numericamente un'approssimazione $\phi^n_j\approx \phi(t_n,x_j)$. Bisogna avere in input i valori
%
\begin{align*}
	V_j^n=V(t_n,x_j),\quad D_j^n=D(t_n,x_j),\quad a_j=a(x_j).
\end{align*}
%
Con questo schema otteniamo l'equazione
%
$$
\frac{\phi^{n+1}_j-\phi^n_j }{\Delta t}+V_j^n\frac{\phi^n_{j+1}-\phi^n_j }{\Delta x}=
D^n_j\frac{\phi^n_{j+1}-2\phi^n_j+\phi^n_{j-1}}{\Delta x^2}+a_j\phi^n_j-b\cdot (\phi^n_j)^2,
$$
%
che scritta in maniera esplicita diventa
%
\begin{align*}
\phi^{n+1}_j=&\quadre{ 1+\Delta t\tonde{ \frac{V_j^n}{\Delta x}-2\frac{D_j^n}{\Delta x^2}+a_j } }\phi^n_j-\Delta t\cdot b\cdot (\phi^n_j)^2\\
&+\Delta t\tonde{ -\frac{V_j^n}{\Delta x}+\frac{D_j^n}{\Delta x^2} }\phi_{j+1}^n+\Delta t\frac{D_j^n}{\Delta x^2}\phi^n_{j-1}.
\end{align*}



\sss{Condizioni al bordo}



Per utilizzare lo schema descritto, servono certamente le condizioni iniziali $\phi_j^0$ per $j=0,\dots, M$, ossia ci serve la funzione $\phi^{(0)}(x)=\phi(0,x)$ assegnata. Inoltre, le formule che descrivono lo schema non sono valide a priori per calcolare $\phi^{n+1}_0$ e $\phi^{n+1}_{M}$, in quanto necessiterebbero dei valori $\phi^{n+1}_{-1}$ e $\phi^{n+1}_{N+1}$ che non sono definiti.

\ssss{Condizioni periodiche }

Un modo per ovviare questo problema è utilizzare la seguente condizione di periodicità:
%
$$
\phi(t,0)=\phi(t,L)\quad \text{per ogni $t\in[0,T]$}.
$$
%
In questo modo si impone $\phi_0^n=\phi_M^n$ per ogni $n$, e si può quindi usare la convenzione $\phi^n_{-1}=\phi^n_{M-1}$ e $\phi^n_{M+1}=\phi^n_1$.


\ssss{Condizioni non periodiche}
Le condizioni periodiche potrebbero non essere un buon modello per determinate situazioni fisiche, o potrebbero dare problemi dal punto di vista computazionale. Per evitarle, bisogna assegnare le condizioni iniziali
%
$$
\phi(t,0)=\phi^{(1)}(t),\quad \phi(t,L)=\phi^{(2)}(t).
$$
In questo modo sono assegnati tutti i valori $\phi^n_0, \phi^n_M$ per $n=0,\dots,N$.
%



\sss{Questioni}


\begin{itemize}
	\item Come scegliere le condizioni al contorno.
	\item Come inizializzare $a(x)$.
	\item Come scegliere i passi $\Delta t, \Delta x$ (forse bisogna far sì che $\Delta x<<\Delta t$).
	\item Estendere lo schema al caso 2d.
\end{itemize}





		

\end{document}


















